\documentclass[12pt]{amsart}
\usepackage{amsfonts, amsmath, latexsym, epsfig}
\usepackage{amssymb}
\usepackage{epsf}
\usepackage{url}


\newcommand{\RR}{\ensuremath{\mathbb{R}}}
\newcommand{\NN}{\ensuremath{\mathbb{N}}}
\newcommand{\QQ}{\ensuremath{\mathbb{Q}}}
\newcommand{\CC}{\ensuremath{\mathbb{C}}}
\newcommand{\ZZ}{\ensuremath{\mathbb{Z}}}
\newcommand{\TT}{\ensuremath{\mathbb{T}}}
\newtheorem{proposition}{Proposition}
\newtheorem{theorem}{Theorem}
\newtheorem{corollary}{Corollary}
\newtheorem{lemma}{Lemma}
\newtheorem{problem}{Problem}
\newtheorem{conjecture}{Conjecture}
\newtheorem{claim}{Claim}
\newtheorem{remark}{Remark}
\newtheorem{definition}{Definition}
%\newcommand{\qed}{\hfill $\Box$ }
%\newcommand{\proof}{\noindent{\bf Proof.}\ \ }
\def\QuotS#1#2{\leavevmode\kern-.0em\raise.2ex\hbox{$#1$}\kern-.1em/\kern-.1em\lower.25ex\hbox{$#2$}}


%\usepackage{vmargin}
%\setpapersize{custom}{21cm}{29.7cm}
%\setmarginsrb{1.7cm}{1cm}{1.7cm}{3.5cm}{0pt}{0pt}{0pt}{0pt}
%marge gauche, marge haut, marge droite, marge bas.
\urlstyle{sf}
%\author{Mathieu DUTOUR SIKIRI\'C}

\DeclareMathOperator{\Aut}{Aut}
\DeclareMathOperator{\Sym}{Sym}


\begin{document}

\author{Mathieu Dutour Sikiri\'c}
\address{Mathieu Dutour Sikiri\'c, Rudjer Boskovi\'c Institute, Bijenicka 54, 10000 Zagreb, Croatia, Fax: +385-1-468-0245}
\email{mathieu.dutour@gmail.com}







\title{Plot\_orientedmap - a program for plotting oriented maps}


\maketitle

\begin{abstract}
  This program allows to plot oriented maps on the sphere or torus and make adequate graphical representations.
  The map may have $1$-gon or $2$-gons. 
\end{abstract}

\section{Designs principles}



\begin{enumerate}
\item {\bf Namelist}: Namelists are used by many other oceanographic and meteorological programs (WWM, COSMO, WAM, etc.) and so the same file format is used by the C++ programs.
\item {\bf Use SVG for representing the graphics}: .svg files are accessible on web browser and can be opened by the inkscape program.
\item {\bf Graphs are represented by oriented edges}: This is a general format that is the most practical for representing oriented maps
\end{enumerate}



\section{Namelist format}

The input of the programs is done in namelist files that contains the information used.
They are formed of blocks of the following kinds.

\begin{verbatim}
&PROC
 MODELNAME = "WWM"
 GridFile = "hgrid.gr3",
 HisPrefix = "WWM_output_",
/
\end{verbatim}

The namelists are done according to following principles:
\begin{enumerate}
\item The beginning of a block is {\tt \&PROC} and the ending is {\tt /}.
\item If a variable is not defined in a block then the default value is used.
\item A variable cannot be defined two times.
\item The blocks and variables can be ordered in anyway the users want. A block may be completely absent in which case all its variables are the default ones.
\item Everything after a {\tt !} is considered as comment and not read.
\item Strings can be delimited by {\tt {\"{ }}} or {\tt '} or nothing, but endings cannot be mixed.
\item When inconsistencies are detected, clear error messages are printed.
\end{enumerate}


\section{Graph representation}

The graphs are represented implicitly by directed edges.
For example if a graph is represented by
\begin{verbatim}
36
 3 6 9 0 12 15 1 18 21 2 24 27 4 22 20 5 28 26 7 30 14 8 13 33 10 31 17 11 16 34 19 25 35 23 29 32
 1 2 0 4 5 3 7 8 6 10 11 9 13 14 12 16 17 15 19 20 18 22 23 21 25 26 24 28 29 27 31 32 30 34 35 33
\end{verbatim}
The number $36$ is the number of directed edges.

The first line lists the inverses of the directed edge, the second line the next directed edge.
For example directed edge numbered $0$ has inverse $3$ and next directed edge $1$. We represent
the corresponding permutations by $\sigma_i$ and $\sigma_n$.

From the directed edges and the permutation we can reconstruct everything:
\begin{enumerate}
\item The edges of the graph correspond to the orbits of the group generated by $\sigma_i$.
  For example $\{0,3\}$ represent one edge of the graph.
\item The vertices of the graph correspond to the orbits of the group generated by $\sigma_n$.
  For example $\sigma_n(0)=1$, $\sigma_n(1)=2$ and $\sigma_n(2)=0$ so $\{0,1,2\}$ represent a vertex
  of the graph.
\item The faces of the graph correspond to the orbits of the group generated by the product permutation $\sigma_i\sigma_n$.
\end{enumerate}

This is a more complicated format than a list of clockwise adjacencies but also more powerful: we can represent $1$-gons, $2$-gons, without any ambiguity. And converting to that format is very easy.


\section{Input file}

The input file has the following sections in it: {\tt PLOT}, {\tt EDGE}, {\tt VERT}, {\tt TORUS}.


Let's see the sections in turn:
\begin{verbatim}
&PLOT
 PlaneFile = "PLOT_1_PL_1_p31m_1_-_1_p3.plane",
 OutFile = "PLOT_1_PL_1_p31m_1_-_1_p3.svg",
 ViewFile = "PLOT_2_PL_6_Oh_2_-_1_S6.view",
 MAX_ITER_PrimalDual = -1,
 MAX_ITER_CaGe = 1000,
 CaGeProcessPolicy = 2,
 RoundMethod = 2,
 width = 600,
 height = 600,
 MethodInsert = 2,
 ListExportFormat = "eps",
/
\end{verbatim}

Signification is following:
\begin{enumerate}
\item {\tt PlaneFile} (string): filename of the file containing the oriented map.
\item {\tt OutFile} (string): filename of the output file.
\item {\tt ViewFile} (string): file of viewfile precising the exterior face. It is needed for planar graph but not for toroidal maps.
\item {\tt MAX\_ITER\_PrimalDual} (integer): maximum number of iteration of primal dual scheme. Put -1 if leaving only after convergence.
\item {\tt MAX\_ITER\_CaGe} (integer): maximum number of iteration of the CaGe scheme.
\item {\tt CaGeProcessPolicy} (integer): policy when activating the CaGe process for toroidal maps. Best to leave it to default value of 2.
\item {\tt RoundMethod} (integer): rounding method used for representing coordinate in .svg file. Leave it at 2.
\item {\tt width/height} (double): width and height in the .svg file. Default value is 600.
\item {\tt MethodInsert} (integer): parameter for selecting the list of representing point. Technical. Leave it at 2.
\item {\tt ListExportFormat} (list of strings): list of selected export formats. Can be {\tt eps}, {\tt png} and {\tt pdf}. This uses inkscape which of course needs to be installed.
\end{enumerate}


Section {\tt EDGE}. Example:
\begin{verbatim}
&EDGE
DoMethod1 = .T.,
DoMethod2 = .F.,
DoMethod3 = .F.,
MultTangent = 0.5,
NormalTraitSize = 1,
ListTraitIDE = 0,3,2,9,4,12,5,15,10,24,13,22,14,20,17,26,19,30,23,33,25,31,32,35,
ListTraitGroup = 0,0,0,0,0,0,0,0,0,0,0,0,0,0,0,0,0,0,0,0,0,0,0,0,
ListTraitSize = 10,
DefaultRGB = 0,0,0,
SpecificRGB_iDE = ,
SpecificRGB_Group = ,
SpecificRGB_R = ,
SpecificRGB_G = ,
SpecificRGB_B = ,
/
\end{verbatim}


Signification is following:
\begin{enumerate}
\item {\tt DoMethod1/DoMethod2/DoMethod3} (logical): best is to leave it at values of T/F/F.
\item {\tt MultTangent} (double): best is to leave at 0.5
\item {\tt NormalTraitSize} (double): trait size of normal edge
\item {\tt DefaultRGB} (list of integer): list of integer values specifying the default color.
\end{enumerate}
Edges can have different colors and also different trait size. This is handled by groups.
In the above example, {\tt ListTraitSize} specifies that we have only one group for which
the trait is $10$.
{\tt ListTraitIDE} specifies the list of directed edge which belongs to specific edges.
{\tt ListTraitGroup} specifies the group to which belong those special directed edges.

Same system is done for the color except that in above example, nothing is specified.



Section {\tt VERT}. It is same as edge except for vertices. Example of section:
\begin{verbatim}
&VERT
 NormalRadius = 0.010,
 ListRadiusIDE = ,
 ListRadiusGroup = ,
 ListRadius = ,
 DefaultRGB = 0,0,0,
 SpecificRGB_iDE = ,
 SpecificRGB_Group = ,
 SpecificRGB_R = ,
 SpecificRGB_G = ,
 SpecificRGB_B = ,
/
\end{verbatim}

This section specifies the radius of the disk around the vertices and the color. Idea are similar
as for edges, that is it works by directed edges.


Section {\tt TORUS}. Example:
\begin{verbatim}
&TORUS
 minimal = 1e-11,
 tol = 1e-05,
 AngDeg = 0,
 scal = 0.5,
 shiftX = 0,
 shiftY = 0,
 FundamentalRGB = 255,0,0,
 FundamentalTraitSize = 2,
 DrawFundamentalDomain = .T.,
/
\end{verbatim}
Those defaults are mostly not to be changed. Remarks:
\begin{enumerate}
\item {\tt minimal/tol} (double): the threshold values used in the primal dual process. If the program does not converge, increase tol and or decrease minimal. Basically, minimal is the square of tol.
\item {\tt scal} (double): the scale factor for representing the map. A factor of $0.5$ means that two fundamental domains are represented which is most often adequate. If it becomes crowded, use a large value like $0.9$.
\item {\tt DrawFundamentalDomain} (logical): sometimes, we want to represent the fundamental domain. If true then the trait size and color can be chosen accordingly.
\end{enumerate}





\end{document}
